For each of the $k$ names, we sample a memory location from $1$ to $n$ with equal probability, with replacement. This is exactly the setup of the birthday problem. Hence, the probability that at least one memory location has more than $1$ value is $$P(A) = 1 - P(A^{c}) = 1 - \frac{n(n-1) \dots (n-k+1)}{n^{k}}$$ Also, $P(A) = 1$ if $n < k$.