Call the two black cards $B_{1}, B_{2}$ and the two red cards $R_{1}, R_{2}$.
Since every configuration of the $4$ cards is equally likely, each outcome has 
a probability of $\frac{1}{24}$ of occurance.

Case 1: $j = 0$.

If both guesses are incorrect, then both of them are black cards. There are 
two choices for the configuration of the black cards and for each, there are 
two choices for the configuration of the red cards for a total of $4$ possibilities.

$$P(j=0) = \frac{4}{24} = \frac{1}{6}$$

Case 2: $j = 4$

Notice that to guess all the cards correctly, we only need to guess correctly 
the two red cards, which, by symmetry, is as likely as guessing both of them wrong.

Hence, $$P(j=4) = P(j=0) = \frac{1}{6}$$

Case 3: $j=2$

One of the guesses is red the other is black. Like before, there are two 
choices for the red and two choices for the black cards. This undercounts the 
possibilities by a factor of 2, since we can switch the places of the red and 
the black cards. Hence, $$P(j=2) = \frac{2}{6} = \frac{1}{3}$$

Notice that getting both right, none right and one right are all the possible 
outcomes. Hence, $$P(j=1) = P(j=3) = 0$$