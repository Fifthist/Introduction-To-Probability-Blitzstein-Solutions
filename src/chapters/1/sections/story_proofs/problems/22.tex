\begin{enumerate}[label=(\alph*)]
\item   Let us count the number of games in a round-robin tournament with $n+1$ 
participants in two ways.
  
  Method 1: Since every player plays against all other players exactly once, 
  the problem reduces to finding the number of ways to pair up $n+1$ people. 
  There are ${n+1 \choose 2}$ ways to do so.
  
  Method 2: The first player participates in $n$ games. The second one also 
  participates in $n$ games, but we have already counted the game against the 
  first player, so we only care about $n-1$ games. The third player also 
  participates in $n$ games, but we have already counted the games against 
  the first and second players, so we only care about $n-2$ games. 
  
  In general, player $i$ will participate in $n+1-i$ games that we care about. 
  Taking the sum over $i$ we get
  
  $$n + (n-1) + (n-2) + \dots + 2 + 1$$
  
  Since both methods count the same thing, they are equal.

\item LHS: If $n$ is chosen first, then the subsequent $3$ numbers can be any 
of $0, 1, \dots, n-1$. These $3$ numbers are chosen with replacement resulting 
in $n^{3}$ possibilities. Summing over possible values of 
$n$ we get $1^{3} + 2^{3} + \dots + n^{3}$ total number of possibilities.
  
  RHS: We can count the number of permutations of the $3$ numbers chosen with 
  replacement from a different perspective. The $3$ numbers can either all be 
  distinct, or all be the same, or differ in exactly $1$ value.
  
  Case 1: All $3$ numbers are distinct.
  
  Selecting $4$ (don't forget the very first, largest selected number) 
  distinct numbers can be done in ${n+1 \choose 4}$ ways. The $3$ smaller 
  numbers are free to permute amongst themselves. This gives us a total of 
  $6{n+1 \choose 4}$ possibilities.
  
  Case 2: All $3$ numbers are the same.
  
  In this case, we have to select $2$ digits. The smaller digit will be sampled 
  $3$ times and there are no ways to permute identical numbers, so the number of 
  possiblities is ${n+1 \choose 2}$.
  
  Case 3: Two of the $3$ numbers are distinct.
  
  In this case, we have to select $3$ digits in total. One of the smaller $2$ 
  digits will be sampled twice, giving us $3$ permutations. Since, there are $2$ 
  choices for which digit gets sampled twice, we get a total of $6$ permutations. 
  The total number of possibilities then is $6{n+1 \choose 3}$.
  
  Adding up the number of possibilities in each of the cases we get a total of 
  $$6{n+1 \choose 4} + 6{n+1 \choose 3} + {n+1 \choose 2}$$ possibilities. 
  
  Since the LHS and the RHS count the same set, they are equal.

\end{enumerate}