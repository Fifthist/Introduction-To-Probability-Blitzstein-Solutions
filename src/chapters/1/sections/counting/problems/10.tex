\begin{enumerate}[label=(\alph*)]
\item Case 1: Student takes exactly one statistics course.
  
  There are $5$ choices for the statistics course. There are ${15 \choose 6}$ 
  choices of selecting $6$ non-statistics courses.
  
  Case 2: Student takes exactly two statistics courses.
  
  There are ${5 \choose 2}$ choices for the two statistics course. 
  There are ${15 \choose 5}$ choices of selecting $5$ non-statistics courses.
  
  Case 3: Student takes exactly three statistics courses.
  
  There are ${5 \choose 3}$ choices for the three statistics course. 
  There are ${15 \choose 4}$ choices of selecting $4$ non-statistics courses.
  
  Case 4: Student takes exactly four statistics courses.
  
  There are ${5 \choose 4}$ choices for the four statistics course. 
  There are ${15 \choose 3}$ choices of selecting $3$ non-statistics courses.
  
  Case 5: Student takes all the statistics courses.
  
  There are ${15 \choose 2}$ choices of selecting $2$ non-statistics courses.
  
  So the total number of choices is
  
  $${5 \choose 1} \times {15 \choose 6} + {5 \choose 2} \times {15 \choose 5} + 
  {5 \choose 3} \times {15 \choose 4} + {5 \choose 4} \times {15 \choose 3} + 
  {5 \choose 5} \times {15 \choose 2}$$

\item It is true that there are ${5 \choose 1}$ ways to select a statistics 
course, and ${19 \choose 6}$ ways to select $6$ more courses from the remaining 
$19$ courses, but this procedure results in overcounting.
  
  For example, consider the following two choices.
  
  \begin{enumerate}[label=(\alph*)]
  \item STAT110, STAT134, History 124, English 101, Calculus 102, Physics 101, Art 121
  \item STAT134, STAT110, History 124, English 101, Calculus 102, Physics 101, Art 121
  \end{enumerate}
  
  Notice that both are selections of the same $7$ courses.
\end{enumerate}