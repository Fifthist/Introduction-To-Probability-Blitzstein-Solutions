Let $F$ be the event that the coin is fair, and let $H_{i}$ be the even that the
$i$-th toss lands Heads.

\begin{enumerate}[label=(\alph*)]
\item Both Fred and his friend are correct. Fred is correct in that the
probability of there being no Heads in the entire sequence is very small. For
example, there are $\binom{92}{45}$ sequences with $45$ Heads and $47$ Tails,
but only $1$ sequence of all Heads.

On the other hand, Fred's friend is correct in his assessment that any
particular sequence has the same likelihood of occurance as any other sequence.

\item 
$$P(F|H_{1 \leq i \leq 92}) = \frac{P(F)P(H_{1 \leq i \leq 92}|F)}{P(F)P(H_{1
\leq i \leq 92}|F) + P(F^{c})P(H_{1\leq i \leq 92}|F^{c})} = \frac{p\left
(\frac{1}{2}\right)^{92}}{p\left(\frac{1}{2}\right)^{92} + (1-p)}$$

\item For $P(F|H_{1 \leq i \leq 92})$ to be larger than $\frac{1}{2}$, $p$ must be
greater than $\frac{2^{92}}{2^{92} + 1}$, which is approximately equal to $1$,
where as for $P(F|H_{1 \leq i \leq 92})$ to be less than $\frac{1}{20}$, $p$
must be less than $\frac{2^{92}}{2^{92} + 19}$, which is also approximately
equal to $1$. In other words, unless we know for a fact that the coin is fair,
$92$ Heads in a row will convince us otherwise.
\end{enumerate}