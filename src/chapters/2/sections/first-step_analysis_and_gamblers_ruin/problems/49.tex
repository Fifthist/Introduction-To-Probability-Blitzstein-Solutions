\begin{enumerate}[label=(\alph*)]
\item
\begin{align*}
P(A_{2}) &= p_{1}p_{2} + q_{1}q_{2} \\
&= (1-q_{1})(1-q_{2}) + \left(b_{1} + \frac{1}{2}\right)\left(b_{2} + \frac{1}
{2}\right) \\
&= \left(b_{1} - \frac{1}{2}\right)\left(b_{2} - \frac{1}{2}\right) + \left(b_
{1} + \frac{1}{2}\right)\left(b_{2} + \frac{1}
{2}\right) \\
&= \frac{1}{2} + 2b_{1}b_{2}
\end{align*}

\item 
By strong induction, $$P(A_{n}) = \frac{1}{2} + 2^{n-1}b_{1}b_{2}...b_{n}$$ for
$n \leq 2.$

Suppose the statement holds for all $n \leq k-1$.
Let $S_{i}$ be the event that the $i$-th trial is a success.

\begin{align*}
P(A_{k}) &= p_{k}P(A_{k-1}^{c} | S_{k}) + q_{k}P(A_{k-1} | S_{k}^{c}) \\
&= p_{k}\left(1 - \left(\frac{1}{2} + 2^{k-2}b_{1}b_{2}...b_{k-1}\right)\right)
+ q_{k}\left(\frac{1}{2} + 2^{k-2}b_{1}b_{2}...b_{k-1}\right) \\
&= p_{k}\left(\frac{1}{2} - 2^{k-2}b_{1}b_{2}...b_{k-1}\right) + q_{k}\left(
\frac{1}{2} + 2^{k-2}b_{1}b_{2}...b_{k-1}\right) \\
&= \frac{1}{2} + (q_{k} - p_{k})2^{k-2}b_{1}b_{2}...b_{k-1} \\
&= \frac{1}{2} + 2b_{k}2^{k-2}b_{1}b_{2}...b_{k-1} \\
&= \frac{1}{2} + 2^{k-1}b_{1}b_{2}...b_{k-1}b_{k}
\end{align*}

\item
if $p_{i} = \frac{1}{2}$ for some $i$, then $b_{i} = 0$ and $P(A_{n}) = \frac{1}
{2}.$

if $p_{i} = 0$ for all $i$, then $b_{i} = \frac{1}{2}$ for all $i$. Hence,
the term $2^{k-1}b_{1}b_{2}...b_{k-1}b_{k}$ equals $\frac{1}{2}$. Thus, $P(A_
{n}) = 1.$ This makes sense since the number of successes will be $0$, which is
an even number.

if $p_{i} = 1$ for all $i$, then $b_{i} = -\frac{1}{2}$ for all $i$. Hence, the
term $2^{k-1}b_{1}b_{2}...b_{k-1}b_{k}$ will either equal to $\frac{1}{2}$ or $-
\frac{1}{2}$ depending on the parity of the number of trials. Thus, $P(A_{n})$
is either $0$ or $1$ depending on the parity of the number of trials.

This makes sense since, if every trial is a success, the number of successes
will be even if the number of trials is even. The number of successes will be
odd otherwise.

\end{enumerate}