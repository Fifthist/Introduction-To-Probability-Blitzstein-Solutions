\begin{enumerate}[label=(\alph*)]

\item Let $B$ be the event that the test done by company $B$ is successfull. Let $A$ be the event that the test done by company $A$ is successfull. Let $D$ be the event that a random person has the disease.

\begin{flalign}
P(B) & = P(D)P(B|D) + P(D^{c})P(B|D^{c}) \nonumber \\
& = 0.01 * 0 + 0.99 * 1 \nonumber \\
& = 0.99 \nonumber
\end{flalign}

\begin{flalign}
P(A) & = P(D)P(A|D) + P(D^{c})P(A|D^{c}) \nonumber \\
& = 0.01 * 0.95 + 0.99 * 0.95 \nonumber \\
& = 0.95 \nonumber
\end{flalign}

Thus, $P(B) > P(A)$.

\item Since the disease is so rare, most people don't have it. Company $B$ diagnoses them correctly every time. However, in the rare cases when a person has the disease, company $B$ fails to diagnose them correctly. Company $A$ however shows a very good probability of an accurate diagnoses for afflicted patients.

\item If the test conducted by company $A$ has equal specifity and sensitivity, then it's accuracy surpasses that of company $B$'s test if the specifity and the sensitivity are larger than $0.99$. If company $A$ manages to achieve a specifity of $1$, then any positive sensitivity will result in a more accurate test. If company $A$ achieves a sensitivity of $1$, it still requires a specificity larger than $0.98$, since positive cases are so rare.
\end{enumerate}