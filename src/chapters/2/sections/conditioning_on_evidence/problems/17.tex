\begin{enumerate}[label=(\alph*)]

\item 
$P(B|A) = \frac{P(B)P(A|B)}{P(B)P(A|B) + P(B^{c})P(A|B^{c})} = 1 
\implies P(B^{c})P(A|B^{c}) = 0$.

Since $P(B^{c}) \neq 0$ by assumption, $P(A|B^{c}) = 0 
\implies P(A^{c}|B^{c}) = 1$.

\item Let $A$ and $B$ be independent events. 
Then, $P(B|A) \approx 1 \implies P(B) \approx 1$.
Thus, $P(B^{c}) \approx 0$, and so the term $P(A|B^{c})$ in the denominator in 
part $a$ may be large,
implying $P(A^{c}|B^{c}) \approx 0$.

For example, consider a deck of $52$ cards, where all but one of the cards are 
the Queen of Spades.
Let $A$ be the event that the first turned card is a Queen of Spades, 
and let $B$ be the event that the second turned card is a Queen of Spades, where
sampling is done with replacement. Then, $P(A) = P(B) \approx 1$.
Then, by independence, $P(A|B^{c}) \approx 1 \implies P(A^{c}|B^{c}) \approx 0$.
\end{enumerate}