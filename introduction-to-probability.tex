\documentclass[]{book}
\usepackage{lmodern}
\usepackage{amssymb,amsmath}
\usepackage{ifxetex,ifluatex}
\usepackage{fixltx2e} % provides \textsubscript
\ifnum 0\ifxetex 1\fi\ifluatex 1\fi=0 % if pdftex
  \usepackage[T1]{fontenc}
  \usepackage[utf8]{inputenc}
\else % if luatex or xelatex
  \ifxetex
    \usepackage{mathspec}
  \else
    \usepackage{fontspec}
  \fi
  \defaultfontfeatures{Ligatures=TeX,Scale=MatchLowercase}
\fi
% use upquote if available, for straight quotes in verbatim environments
\IfFileExists{upquote.sty}{\usepackage{upquote}}{}
% use microtype if available
\IfFileExists{microtype.sty}{%
\usepackage[]{microtype}
\UseMicrotypeSet[protrusion]{basicmath} % disable protrusion for tt fonts
}{}
\PassOptionsToPackage{hyphens}{url} % url is loaded by hyperref
\usepackage[unicode=true]{hyperref}
\hypersetup{
            pdftitle={Introduction to Probability, Second Edition},
            pdfauthor={Fifthist},
            pdfborder={0 0 0},
            breaklinks=true}
\urlstyle{same}  % don't use monospace font for urls
\usepackage{natbib}
\bibliographystyle{apalike}
\usepackage{longtable,booktabs}
% Fix footnotes in tables (requires footnote package)
\IfFileExists{footnote.sty}{\usepackage{footnote}\makesavenoteenv{long table}}{}
\usepackage{graphicx,grffile}
\makeatletter
\def\maxwidth{\ifdim\Gin@nat@width>\linewidth\linewidth\else\Gin@nat@width\fi}
\def\maxheight{\ifdim\Gin@nat@height>\textheight\textheight\else\Gin@nat@height\fi}
\makeatother
% Scale images if necessary, so that they will not overflow the page
% margins by default, and it is still possible to overwrite the defaults
% using explicit options in \includegraphics[width, height, ...]{}
\setkeys{Gin}{width=\maxwidth,height=\maxheight,keepaspectratio}
\IfFileExists{parskip.sty}{%
\usepackage{parskip}
}{% else
\setlength{\parindent}{0pt}
\setlength{\parskip}{6pt plus 2pt minus 1pt}
}
\setlength{\emergencystretch}{3em}  % prevent overfull lines
\providecommand{\tightlist}{%
  \setlength{\itemsep}{0pt}\setlength{\parskip}{0pt}}
\setcounter{secnumdepth}{5}
% Redefines (sub)paragraphs to behave more like sections
\ifx\paragraph\undefined\else
\let\oldparagraph\paragraph
\renewcommand{\paragraph}[1]{\oldparagraph{#1}\mbox{}}
\fi
\ifx\subparagraph\undefined\else
\let\oldsubparagraph\subparagraph
\renewcommand{\subparagraph}[1]{\oldsubparagraph{#1}\mbox{}}
\fi

% set default figure placement to htbp
\makeatletter
\def\fps@figure{htbp}
\makeatother

\usepackage{booktabs}
\usepackage{amsthm}
\makeatletter
\def\thm@space@setup{%
  \thm@preskip=8pt plus 2pt minus 4pt
  \thm@postskip=\thm@preskip
}
\makeatother

\title{\emph{Introduction to Probability, Second Edition}}
\author{Fifthist}
\date{2020-03-20}

\begin{document}
\maketitle

{
\setcounter{tocdepth}{1}
\tableofcontents
}
\chapter*{Preface}\label{preface}
\addcontentsline{toc}{chapter}{Preface}

This book is an unofficial solution manual for the exercises in
\emph{Introduction to Probability, Second Edition} by Joseph Blitzstein
and Jessica Hwang.

\chapter{Probability and Counting}\label{probability-and-counting}

\section{Counting}\label{counting}

\subsection{}\label{section}

\textbf{Intuition}

Imagine eleven empty slots to place the letters into.

How many ways are there to place the four \(I\)-s into the slots? For
each placement of \(I\)s, can we figure out the number of ways to place
the remaining letters into the \(7\) empty slots?

 \textbf{Solution}

We have one \(M\), four \(I\)-s, four \(S\)-s, and two \(P\)-s. There
are \({11 \choose 4}\) ways to place the \(I\)-s, \({7 \choose 4}\) ways
to place \(S\)-s, \({3 \choose 2}\) ways to place the \(P\)-s, and
\({1 \choose 1}\) ways to place the \(M\).

\[ {11 \choose 4} \times {7 \choose 4} \times {3 \choose 2} \times {1 \choose 1} \]

\subsection{}\label{section-1}

a.~\textbf{Intuition}

If the first digit can't be \(0\) or \(1\), how many choices are we left
with for the first digit? For each choice of first digit, how many
choices do we have for the remaining six digits?

 \textbf{Solution}

If the first digit can't be \(0\) or \(1\), we have eight choices for
the first digit - \(2\) to \(9\). The remaining six digits can be
anything from \(0\) to \(9\). Hence, the solution is
\[ 8 \times 10^{6} \]

b.~\textbf{Intuition}

How many phone numbers start with \(911\)?

Can we use the answer from the previous part to find the desired
quantity?

 \textbf{Solution}

We can subtract the number of phone numbers that start with \(911\) from
the total number of phone numbers we found in the previous part.

If a phone number starts with \(911\), it has ten choices for each of
the remaining four digits.

\[ 8 \times 10^{6} - 10^{4} \]

\subsection{}\label{section-2}

a.~\textbf{Intuition}

How many choices of restaurants does Fred have on Monday?

Once Fred attends a restaurant on Monday, how many choices of
restaurants does he have for the remainder of the week?

 \textbf{Solution}

Fred has \(10\) choices for Monday, \(9\) choices for Tuesday, \(8\)
choices for Wednesday, \(7\) choices for Thursday and \(6\) choices for
Friday.

\[ 10 \times 9 \times 8 \times 7 \times 6 \]

b.~\textbf{Intution}

We are told that Fred will not attend a restaurant he went to the
previous day, but can he go to a restaurant he went to two or more days
ago?

 \textbf{Solution}

For the first restaurant, Fred has \(10\) choices. For all subsequent
days, Fred has \(9\) choices, since the only restriction is that he
doesn't want to eat at the restaurant he ate at the previous day.

\[ 10 \times 9^{4} \]

\subsection{}\label{section-3}

a.~\textbf{Intuition}

How many matches are there in a \emph{round-robin} tournament?

How many outcomes are possible for each match?

 \textbf{Solution}

There are \({n \choose 2}\) matches.

For a given match, there are two outcomes. Each match has two possible
outcomes. We can use the multiplication rule to count the total possible
outcomes.

\[ 2^{{n \choose 2}} \]

b.~\textbf{Intuition}

How many opponents will every player play against?

How many times will a given pair of players face each other?

 \textbf{Solution}

Since every player plays every other player exactly once, the number of
games is the number of ways to pair up \(n\) people.

\[ {n \choose 2} \]

\subsection{}\label{section-4}

a.~\textbf{Intuition}

How many players are left by the end of a round compared to the number
of players at the start of the round?

How many rounds need to pass for a single player to be left standing?

 \textbf{Solution}

By the end of each round, half of the players participating in the round
are eliminated. So, the problem reduces to finding out how many times
the number of players can be halved before a single player is left.

The number of times \(N\) can be divided by two is \[\log_{2}{N}\]

b.~\textbf{Intuition}

Suppose there are \(N_{r}\) players at the start of round \(r\). If
every player plays exactly one game, how many games will be played in
round \(r\)?

 \textbf{Solution}

The number of games in a given round is \(\frac{N_{r}}{2}\). We can sum
up these values for all the rounds.

\begin{equation} 
  \begin{split}
   f(N) & = \frac{N}{2} + \frac{N}{4}  + \frac{N}{8} + \dots + \frac{N}{2^{\log_{2}{N}}}\\
   & =N \sum_{i=0}^{\log_{2}{N}} \frac{1}{2^{i}}\\
   & =N \times \frac{N-1}{N}\\
   & =N-1
  \end{split}
  \end{equation}

c.~\textbf{Intuition}

How many players need to be eliminated before the tournament is over?

How many players are eliminated as a result of a single match?

 \textbf{Solution}

Tournament is over when a single player is left. Hece, \(N-1\) players
need to be eliminated. As a result of a match, exactly one player is
eliminated. Hence, the number of matches needed to eliminate \(N-1\)
people is

\[ N-1 \]

\subsection{}\label{section-5}

\textbf{Intuition}

How many ways can we match up twenty chess players if we don't care
about who plays with white and who plays with black pieces?

Can we use the answer from the previous part to find the desired
quantity?

 \textbf{Solution}

There are \({20 \choose 2}\) ways to pair up twenty chess players. For
each pairing, we can first let player \(A\) play with whites, then let
player \(B\) play with whites. Thus, for each of the \({20 \choose 2}\)
pairs, we have \(2\) matches for a total of

\[ {20 \choose 2} \times 2 \] matches.

\subsection{}\label{section-6}

a.~\textbf{Intuition}

How many ways are there to assign three wins to player \(A\)?

Out of the remaining four games, how many ways are there to assign two
draws and two losses to \(A\)?

 \textbf{Solution}

There are \({7 \choose 3}\) ways to assign three wins to player \(A\).
For a specific combination of three games won by \(A\), there are
\({4 \choose 2}\) ways to assign two draws to \(A\). There is only one
way to assign two losses to \(A\) from the remaining two games, namely,
\(A\) losses both games.

\[ {7 \choose 3} \times {4 \choose 2} \times {2 \choose 2} \]

b.~\textbf{Intuition}

Can \(A\) get \(4\) points if \(A\) never wins? What if \(A\) wins more
than \(4\) games?

List the possible outcomes of games that award \(4\) points to \(A\).

 \textbf{Solution}

If \(A\) were to draw every game, there would need to be at least \(8\)
games for \(A\) to obtain \(4\) points, so \(A\) has to win at least
\(1\) game. Similarly, if \(A\) wins more than \(4\) games, they will
have more than \(4\) points.

Case 1: \(A\) wins \(1\) game and draws \(6\).

This case amounts to selecting \(1\) out of \(7\) for \(A\) to win and
assigning a draw for the other \(6\) games. Hence, there are \(7\)
possibilities.

Case 2: \(A\) wins \(2\) games and draws \(4\).

There are \({7 \choose 2}\) ways to assign \(2\) wins to \(A\). For each
of them, there are \({5 \choose 4}\) ways to assign four draws to \(A\)
out of the remaining \(5\) games. Player \(B\) wins the remaining game.
The total number of possibilites for this case is
\({7 \choose 2} \times {5 \choose 4}\).

Case 3: \(A\) wins \(3\) games and draws \(2\).

There are \({7 \choose 3}\) ways to assign \(3\) wins to \(A\). For each
of them, there are \({4 \choose 2}\) ways to assign two draws to \(A\)
out of the remaining \(4\) games. \(B\) wins the remaining \(2\) games.
The total number of possibilites for this case is
\({7 \choose 3} \times {4 \choose 2}\).

Case 4: \(A\) wins \(4\) games and loses \(3\).

There are \({7 \choose 4}\) ways to assign \(4\) wins to \(A\). \(B\)
wins the remaining \(3\) games. The total number of possibilites for
this case is \({7 \choose 4}\).

Summing up the number of possibilities in each of the cases we get

\[ {7 \choose 1} + {7 \choose 2} \times {5 \choose 4} + {7 \choose 3} \times {4 \choose 2} + {7 \choose 4} \]

c.~\textbf{Intuition}

Given the final score of \(4\) to \(3\) and the fact that the match will
end if either of the players reaches \(4\) points, could \(B\) have been
the player to win the last game?

Suppose \(A\) wins the last game. Could \(A\) have won only \(1\) game
out of the first \(6\)?

Count the number of possibilites for the case when \(A\) wins the last
game and the number of possibilities for the case when \(A\) draws the
last game.

 \textbf{Solution}

If \(B\) were to win the last game, that would mean that \(A\) had
already obtained \(4\) points prior to the last game, so the last game
would not be played at all. Hence, \(B\) could not have won the last
game.

Case 1: \(A\) wins \(3\) out of the first \(6\) games and wins the last
game.

There are \({6 \choose 3}\) ways to assign \(3\) wins to \(A\) out of
the first \(6\) games. The other \(3\) games end in a draw. The number
of possibilities then is \({6 \choose 3}\).

Case 2: \(A\) wins \(2\) and draws \(2\) out of the first \(6\) games
and wins the last game.

There are \({6 \choose 2}\) ways to assign \(2\) wins to \(A\) out of
the first \(6\) games. From the \(4\) remaining games, there are
\({4 \choose 2}\) ways to assign \(2\) draws. The remaining \(2\) games
are won by \(B\). The number of possibilities is
\({6 \choose 2} \times {4 \choose 2}\).

Case 3: The last game ends in a draw.

This case implies that \(A\) had \(3.5\) and \(B\) had \(2.5\) points by
the end of game \(6\).

Case 3.1: \(A\) wins \(3\) and draws \(1\) out of the first \(6\) games.

There are \({6 \choose 3}\) ways to assign \(3\) wins to \(A\) out of
the first \(6\) games. There are \({3 \choose 1}\) ways to assign a draw
out of the remaining \(3\) games. \(B\) wins the other \(2\) games. The
number of possibilities is \({6 \choose 3} \times {3 \choose 1}\).

Case 3.2: \(A\) wins \(2\) and draws \(3\) out of the first \(6\) games.

There are \({6 \choose 2}\) ways to assign \(2\) wins to \(A\) out of
the first \(6\) games. There are \({4 \choose 3}\) ways to assign \(3\)
draws out of the remaining \(4\) games. \(B\) wins the remaining game.
The number of possibilities is \({6 \choose 2} \times {4 \choose 3}\).

Case 3.3: \(A\) wins \(1\) and draws \(5\) of the first \(6\) games.

There are \({6 \choose 1}\) ways to assign a win to \(A\) out of the
first \(6\) games.

The total number of possibilities then is

\[ {6 \choose 3} + {6 \choose 2} \times {4 \choose 2} + {6 \choose 3} \times {3 \choose 1} + {6 \choose 2} \times {4 \choose 3} + {6 \choose 1}\]

\subsection{}\label{section-7}

Solution is provided by the author.

\subsection{}\label{section-8}

Solution is provided by the author.

\subsection{}\label{section-9}

a.~\textbf{Intuition}

How many choices are there if the student takes only one statistics
course? What about two statistics courses?

 \textbf{Solution}

Case 1: Student takes exactly one statistics course.

There are \(5\) choices for the statistics course. There are
\({15 \choose 6}\) choices of selecting \(6\) non-statistics courses.

Case 2: Student takes exactly two statistics courses.

There are \({5 \choose 2}\) choices for the two statistics course. There
are \({15 \choose 5}\) choices of selecting \(5\) non-statistics
courses.

Case 3: Student takes exactly three statistics courses.

There are \({5 \choose 3}\) choices for the three statistics course.
There are \({15 \choose 4}\) choices of selecting \(4\) non-statistics
courses.

Case 4: Student takes exactly four statistics courses.

There are \({5 \choose 4}\) choices for the four statistics course.
There are \({15 \choose 3}\) choices of selecting \(3\) non-statistics
courses.

Case 5: Student takes all the statistics courses.

There are \({15 \choose 2}\) choices of selecting \(2\) non-statistics
courses.

So the total number of choices is

\[{5 \choose 1} \times {15 \choose 6} + {5 \choose 2} \times {15 \choose 5} + {5 \choose 3} \times {15 \choose 4} + {5 \choose 4} \times {15 \choose 3} + {5 \choose 5} \times {15 \choose 2}\]

b.~\textbf{Intuition}

Would \({5 \choose 1} \times {19 \choose 6}\) overcount any choices?

 \textbf{Solution}

It is true that there are \({5 \choose 1}\) ways to select a statistics
course, and \({19 \choose 6}\) ways to select \(6\) more courses from
the remaining \(19\) courses, but this procedure results in
overcounting.

For example, consider the following two choices.

\begin{enumerate}
\def\labelenumi{\alph{enumi}.}
\tightlist
\item
  STAT110, STAT134, History 124, English 101, Calculus 102, Physics 101,
  Art 121
\item
  STAT134, STAT110, History 124, English 101, Calculus 102, Physics 101,
  Art 121
\end{enumerate}

Notice that both are selections the same \(7\) courses.

\subsection{}\label{section-10}

a.~\textbf{Intuition}

To specify a function, we need to assign an output to every input. How
many ways are there to do this?

 \textbf{Solution}

Each of the \(n\) inputs has \(m\) choices for an output, resulting in
\[m^{n}\] possible functions.

b.~\textbf{Intuition}

A function is \emph{one-to-one} if it maps unique inputs to unique
outputs. Suppose \(n < m\). Can a function be one-to-one?

 \textbf{Solution}

If \(n < m\), at least two inputs will be mapped to the same output, so
no one-to-one function is possible.

If \(n \geq m\), the first input has \(m\) choices, the second input has
\(m - 1\) choices, and so on. The total number of one-to-one functions
then is \[m(m-1)(m-2)\dots(m-n+1)\]

\subsection{}\label{section-11}

a.~\textbf{Intuition}

How many ways are there to select \(13\) cards out of a standard deck?

 \textbf{Solution}

\[{52 \choose 13}\]

b.~\textbf{Intuition}

How many ways are there to break a standard deck into \(4\) groups of
size \(13\)? Can we use this result to get the desired quantity?

 \textbf{Solution}

The number of ways to break \(52\) cards into \(4\) groups of size
\(13\) is
\[\frac{{52 \choose 13}{39 \choose 13}{26 \choose 13}{13 \choose 13}}{4!}\].

The reason for dividision by \(4!\) is that all permutations of specific
\(4\) groups describe the same way to group \(52\) cards.

Since we do care about the order of the \(4\) groups, we should not
divide by \({4!}\). The final answer then is
\[{52 \choose 13}{39 \choose 13}{26 \choose 13}{13 \choose 13}\]

c.~\textbf{Intuition}

Does dealing \(13\) cards to player \(A\) change the number of possible
hands \(B\) could get?

 \textbf{Solution}

The key is to notice that the sampling is done \emph{without
replacement}. \({52 \choose 13}^{4}\) assumes that all four players have
\({52 \choose 13}\) choices of hands available to them. This would be
true if sampling was done \emph{with replacement}.

\subsection{}\label{section-12}

\textbf{Intuition}

Why is the answer not \({520 \choose 10}\)?

Why is the answer not \(520^{10}\)?

 \textbf{Solution}

The problem amounts to sampling with replacement where order does not
matter, since having \(10\) copies of each card amounts to replacing the
card. This is done using the Bose-Einstein method.

Thus, the answer is \[{52 + 10 - 1 \choose 10} = {61 \choose 10}\]

\subsection{}\label{section-13}

\textbf{Intuition}

How many possibilities are there for a small pizza? What about a medium
pizza?

Note that ordering a Small Vegetarian and a Large Pepperoni is the same
as ordering a Large Pepperoni and a Small Vegetarian.

 \textbf{Solution}

There are \(4\) choices for sizes and \(9\) choices for toppings for a
total of \(36\) possibilities for a single pizza. Since we are ordering
\(2\) pizzas, we get \(36^{2}\) possibilities. However, this result
overcounts the desired quantity by a factor of \(2\), since

\begin{enumerate}
\def\labelenumi{\arabic{enumi}.}
\tightlist
\item
  Small Vegetarian, Large Pepperoni
\item
  Large Pepperoni, Small Vegetarian
\end{enumerate}

describe the same order. The number of possibilities then is
\[\frac{36^{2}}{2}\]

\section{Story Proofs}\label{story-proofs}

\subsection{}\label{section-14}

Solution provided by the author

\subsection{}\label{section-15}

Solution provided by the author

\subsection{}\label{section-16}

\textbf{Intuition}

How many ways are there to sample \(n\) objects from a set of \(2n\)?

Can we break up the original set into two sets of size \(n\) and sample
from both sets?

 \textbf{Solution}

\({2n \choose n}\) counts the number of ways to sample \(n\) objects
from a set of \(2n\). Instead of sampling from the whole set, we can
break the set into two sets of size \(n\) each. Then, we have to sample
\(n\) objects in total from both sets.

We can sample all \(n\) objects from the first set, or \(1\) object from
the first set and \(n-1\) objects from the second set, or \(2\) objects
from the first set and \(n-2\) objects from the second set and so on.

There are \({n \choose n}\) ways to sample all \(n\) objects from the
first set, \({n \choose 1} {n \choose n-1}\) ways to sample \(1\) object
from the first set and \(n-1\) objects from the second set,
\({n \choose 2} {n \choose n-2}\) ways to sample \(2\) objects from the
first set and \(n-2\) objects from the second set. The pattern is clear
\[ \sum_{k=0}^{n} {n \choose k} {n \choose n-k} = \sum_{k=0}^{n} {n \choose k}^{2}\]

\subsection{}\label{section-17}

\textbf{Intuition}

If a person is selected for a committee chair, how many ways are there
to select the remaining members?

If \(m\) members are selected from the first group and \(n-m\) members
are selected from the second group, how many ways are there to assign a
chair?

 \textbf{Solution}

Consider the right hand side of the equation. Since a committe chair can
only be selected from the first group, there are \(n\) ways to choose
them. Then, for each choice of a committee chair, there are
\({2n-1 \choose n-1}\) ways to choose the remaining members. Hence, the
total number of committees is \(n{2n-1 \choose n-1}\).

Now consider the left side of the equation. Suppose we pick \(k\) people
from the first group and \(n-k\) people from the second group, then
there are \(k\) ways to assign a chair from the members of the first
group we have picked. \(k\) can range from \(1\) to \(n\) giving us a
total of
\(\sum_{k=1}^{n} k{n \choose k}{n \choose n-k} = \sum_{k=1}^{n} k{n \choose k}^{2}\)
possible committees.

Since, both sides of the equation count the same thing, they are equal.

\subsection{}\label{section-18}

\textbf{Intuition}

Expanding on the hint provided by the author, think of elements in the
subsets as ordered from lowest to largest. What is the smallest and the
largest values a middle element can have?

If the middle element is \(k+1\), how many choices do we have for the
left half of the subset? What about the right half?

 \textbf{Solution}

Since the subsets have size \(5\), a middle element can range from \(3\)
to \(n+1\). Let us label middle elements as \(k+1\). Then, there are
\({k \choose 2}\) choises of elements for the left half of a subset and
\({n+3-(k+1) \choose 2} = {n+2-k \choose 2}\) choices for the right
half.

Taking the sum as \(k+1\) ranges from \(3\) to \(n+1\), we get the
desired result.

\subsection{}\label{section-19}

Solution provided by the author

\subsection{}\label{section-20}

a.~\textbf{Intuition}

Suppose there are \(n+1\) people at a party, and we need to group them
into \(k\) non empty groups. Let us focus on a specific person. Call him
Tony.

Suppose Tony is in a group by himself. How many ways are there to group
the remaining \(n\) people?

If Tony is not in a group by himself, how many groups can he join?

 \textbf{Solution}

Case 1: If Tony is in a group by himself, then we have to break the
remaining \(n\) people into \(k-1\) groups. This can be done in \[
  \left\{
    \begin{array}{cc}
        n\\
        k-1
    \end{array}
  \right\}
  \] ways.

Case 2: If Tony is not in a group by himself, then we first break up the
remaining \(n\) people into \(k\) groups. Then, Tony can join any of
them. The number of possible groups then is \[
  k\left\{
    \begin{array}{cc}
        n\\
        k
    \end{array}
  \right\}
  \]

Case 1 and 2 together count the number of ways to break up \(n+1\)
people into \(k\) non empty groups, which is precisely what the left
side of the equation counts.

b.~\textbf{Intuition}

Suppose we would like to break the \(n+1\) guests into \(k+1\) groups
instead of \(k\).

Tony wants to have \(3\) people in his group. How many ways are there to
select them? For each such selection, how many ways are there to group
the remaining people?

What is the largest number of people Tony can have in his group?

 \textbf{Solution}

Say Tony wants to have \(m\) in his group. That is to say he does not
want \(n-m\) people. These \(n-m\) people must then be broken into \(k\)
groups.

The number of people Tony wants to join his group can range from \(0\)
to \(n-k\). The reason for the upper bound is that at least \(k\) people
are required to make up the remaining \(k\) groups.

Taking the sum over the number of people in Tony's group we get
\[\sum_{j=0}^{n-k}{n \choose j}\left\{
    \begin{array}{cc}
        n-j\\
        k
    \end{array}
  \right\}
  \]

Now, instead of taking the sum over the number of people Tony wants in
his group, we can equivalently take the sum over the number of people
Tony does not want in his group. Hence,

\[\sum_{j=0}^{n-k}{n \choose j}\left\{
      \begin{array}{cc}
          n-j\\
          k
      \end{array}
    \right\} = \sum_{i=n}^{k}{n \choose i}\left\{
      \begin{array}{cc}
          i\\
          k
      \end{array}
    \right\}
  \] Since the sum counts all possible ways to group \(n+1\) people into
\(k+1\) groups, we have \[
\sum_{i=n}^{k}{n \choose i}\left\{
      \begin{array}{cc}
          i\\
          k
      \end{array}
    \right\} = 
    \left\{
      \begin{array}{cc}
          n+1\\
          k+1
      \end{array}
    \right\}
\] as desired.

\subsection{}\label{section-21}

a.~\textbf{Intuition}

How many games are there in a round-robin tournament with \(n+1\)
participants?

What is wrong with the following line of thought? \(3\) people
participate in a round-robin tournament. Player \(A\) participates in
\(2\) games. So do players \(B\) and \(C\). Hence, the total number of
games in the tournament is \(2 + 2 + 2 = 6\).

 \textbf{Solution}

Let us count the number of games in a round-robin tournament with
\(n+1\) participants in two ways.

Method 1: Since every player plays against all other players exactly
once, the problem reduces to finding the number of ways to pair up
\(n+1\) people. There are \({n+1 \choose 2}\) ways to do so.

Method 2: The first player participates in \(n\) games. The second one
also participates in \(n\) games, but we have already counted the game
against the first player, so we only care about \(n-1\) games. The third
player also participates in \(n\) games, but we have already counted the
games against the first and second players, so we only care about
\(n-2\) games.

In general, player \(i\) will participate in \(n+1-i\) games that we
care about. Taking the sum over \(i\) we get

\[n + (n-1) + (n-2) + \dots + 2 + 1\]

Since both methods count the same thing, they are equal.

b.~\textbf{Intuition}

Consider the scenario hinted by the author.

If a person picks the number \(7\), how many choices do they have for
the next \(3\) numbers? Can we generalize the result?

Now we consider the right side of the equation. Suppose the \(3\)
numbers a person has chosen with replacement are distinct, how many
possible permutations are there? What if \(2\) of the numbers are
distinct? What if none of them are distinct?

 \textbf{Solution}

LHS: If \(n\) is chosen first, then the subsequent \(3\) numbers can be
any of \(0, 1, \dots, n-1\). These \(3\) numbers are chosen with
replacement resulting in \(n^{3}\) possibilities. Summing over possible
values of \(n\) we get \(1^{3} + 2^{3} + \dots + n^{3}\) total number of
possibilities.

RHS: We can count the number of permutations of the \(3\) numbers chosen
with replacement from a different perspective. The \(3\) numbers can
either all be distinct, or all be the same, or differ in exactly \(1\)
value.

Case 1: All \(3\) numbers are distinct.

Selecting \(4\) (don't forget the very first, largest selected number)
distinct numbers can be done in \({n+1 \choose 4}\) ways. The \(3\)
smaller numbers are free to permute amongst themselves. This gives us a
total of \(6{n+1 \choose 4}\) possibilities.

Case 2: All \(3\) numbers are the same.

In this case, we have to select \(2\) digits. The smaller digit will be
sampled \(3\) times and there are no ways to permute identical numbers,
so the number of possiblities is \({n+1 \choose 2}\).

Case 3: Two of the \(3\) numbers are distinct.

In this case, we have to select \(3\) digits in total. One of the
smaller \(2\) digits will be sampled twice, giving us \(3\)
permutations. Since, there are \(2\) choices for which digit gets
sampled twice, we get a total of \(6\) permutations. The total number of
possibilities then is \(6{n+1 \choose 3}\).

Adding up the number of possibilities in each of the cases we get a
total of \[6{n+1 \choose 4} + 6{n+1 \choose 3} + {n+1 \choose 2}\]
possibilities.

Since the LHS and the RHS count the same set, they are equal.

\section{Naive definition of
probability}\label{naive-definition-of-probability}

\subsection{}\label{section-22}

\textbf{Intuition}

Suppose all three people are going to the same floor, how many choice
are there?

What if only two of them are going to the same floor? What if they are
all going to different floors?

 \textbf{Solution}

Case 1: All three go to the same floor.

They can go to floors \(2\) to \(10\) for a total of \(9\) choices.

Case 2: Two of them go to the same floor.

There are \(9 \choose 2\) possibilities in this case.

Case 3: They all go to different floors.

There are \(9 \choose 3\) possibilities in this case.

We are interested in the case of \(3\) consecutive floors. There are
\(7\) equally likely possibilities
\[(2, 3, 4), (3, 4, 5), (4, 5, 6), (5, 6, 7), (6, 7, 8), (7, 8, 9), (8, 9, 10).\]

Hence, the probability that the buttons for \(3\) consecutive floors are
pressed is \[\frac{7}{9 + {9 \choose 2} + {9 \choose 3}}\]

\bibliography{book.bib,packages.bib}

\end{document}
